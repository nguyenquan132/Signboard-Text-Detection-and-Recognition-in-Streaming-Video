\chapter{PHƯƠNG PHÁP}
\ifpdf
    \graphicspath{{Chapter3/Chapter3Figs/PNG/}{Chapter3/Chapter3Figs/PDF/}{Chapter3/Chapter3Figs/}}
\else
    \graphicspath{{Chapter3/Chapter3Figs/EPS/}{Chapter3/Chapter3Figs/}}
\fi

\section{Hệ thống phát hiện và nhận dạng chữ trên biển hiệu}

\subsection{Bài toán và mục tiêu}
Mục tiêu của khóa luận là xây dựng một hệ thống tự động ``đọc chữ trên biển hiệu'' trong ảnh/video đường phố.
Dữ liệu đầu vào có thể là ảnh tĩnh hoặc chuỗi khung hình video; đầu ra là (i) vị trí biển hiệu, (ii) vị trí vùng chữ trên biển hiệu, và (iii) nội dung văn bản được nhận dạng.

Do đặc trưng dữ liệu street-view chứa nhiều nhiễu nền và văn bản đa dạng về hình dạng, hướng, kích thước, nhóm đề xuất một pipeline dạng module hoá gồm ba thành phần chính:
\begin{itemize}
    \item \textbf{Phát hiện biển hiệu (Signboard Detection/Segmentation):} khoanh vùng biển hiệu để giảm không gian tìm kiếm và loại nhiễu nền.
    \item \textbf{Phát hiện văn bản (Text Detection):} phát hiện vùng chữ bên trong biển hiệu (ưu tiên OBB/đa giác khi chữ nghiêng).
    \item \textbf{Nhận dạng văn bản (Text Recognition):} nhận dạng chuỗi ký tự từ vùng chữ đã crop/rectify.
\end{itemize}

\subsection{Kiến trúc tổng thể của pipeline}
Hình~\ref{fig:ch3_pipeline_overall} minh họa luồng xử lý tổng thể. Pipeline được thiết kế theo hướng ``tách nhiệm vụ'' để linh hoạt thay thế mô-đun (detector/recognizer) tùy điều kiện tài nguyên hoặc yêu cầu tốc độ.

\begin{figure}[t]
    \centering
    \includegraphics[width=0.98\linewidth]{pipeline_overall.png}
    \caption{Kiến trúc tổng thể hệ thống đọc chữ trên biển hiệu: Signboard Detection/Segmentation $\rightarrow$ (Align/Rectify) $\rightarrow$ Text Detection $\rightarrow$ Text Recognition $\rightarrow$ Post-processing.}
    \label{fig:ch3_pipeline_overall}
\end{figure}

Với mỗi khung hình/ảnh đầu vào $I$, hệ thống thực hiện:
\begin{enumerate}
    \item Dự đoán vùng biển hiệu $\mathcal{S} = \{s^{(i)}\}$ bằng mô hình detection (bbox/OBB) hoặc segmentation (mask).
    \item Chuẩn hoá hình học biển hiệu (tuỳ chọn) để giảm méo phối cảnh, thu được ảnh biển hiệu đã crop: $I_{sb}^{(i)}$.
    \item Chạy Text Detector trên $I_{sb}^{(i)}$ để lấy tập vùng chữ $\mathcal{B}=\{b^{(j)}\}$.
    \item Với mỗi $b^{(j)}$, thực hiện crop \& rectify để tạo patch chữ $\hat{I}^{(j)}$.
    \item Chạy Text Recognizer để dự đoán chuỗi ký tự $\mathbf{s}^{(j)}$ và độ tin cậy $c^{(j)}$.
    \item Hậu xử lý (chuẩn hoá Unicode, lọc theo confidence, gộp theo dòng/khối nếu cần).
\end{enumerate}

\subsection{Quy ước biểu diễn hình học (BBox/OBB/Mask)}
Trong khóa luận, để mô tả vùng quan tâm (biển hiệu hoặc chữ), nhóm sử dụng một trong ba dạng:
\begin{itemize}
    \item \textbf{Rectangle bounding box (BBox):} $b=(x_{min}, y_{min}, x_{max}, y_{max})$.
    \item \textbf{Oriented bounding box (OBB):} hộp quay, biểu diễn bởi (i) 4 đỉnh hoặc (ii) $(x_c, y_c, w, h, \theta)$.
    \item \textbf{Segmentation mask:} mặt nạ nhị phân $m(x,y)\in\{0,1\}$.
\end{itemize}
OBB/mask đặc biệt hữu ích khi biển hiệu hoặc chữ bị nghiêng/biến dạng do phối cảnh.

\subsection{Chiến lược ``phát hiện biển hiệu trước''}
Thay vì phát hiện chữ trực tiếp trên toàn ảnh street-view (nhiều nhiễu), nhóm lựa chọn chiến lược:
\[
I \xrightarrow[]{\text{Signboard module}} \{I_{sb}^{(i)}\} \xrightarrow[]{\text{Text detection/recognition}} \text{outputs}
\]
Lợi ích chính:
\begin{itemize}
    \item \textbf{Giảm nhiễu nền:} hạn chế text detector bị đánh lạc hướng bởi biển số xe, poster nhỏ, vật thể nền.
    \item \textbf{Giảm chi phí tính toán:} text detector/recognizer chạy trên patch nhỏ thay vì ảnh full-HD.
    \item \textbf{Tăng độ ổn định:} vùng biển hiệu thường chứa text liên quan, giúp mô hình tập trung đúng ngữ cảnh.
\end{itemize}

% -------------------------------------------------
\section{Mô-đun phát hiện biển hiệu (Signboard Detection/Segmentation)}

\subsection{Phát hiện biển hiệu bằng object detection}
Nhóm thực nghiệm fine-tune các mô hình object detection hiện đại trên SignboardText để dự đoán vị trí biển hiệu dưới dạng hộp chữ nhật (BBox) hoặc hộp quay (OBB).
Về mặt phương pháp, object detector thực hiện ánh xạ:
\[
f_{\text{det}}(I) \rightarrow \{(b^{(i)}, p^{(i)})\}
\]
trong đó $b^{(i)}$ là bbox/obb, $p^{(i)}$ là confidence.

\subsubsection{Lý do ưu tiên OBB cho biển hiệu}
Biển hiệu trong street-view thường bị nghiêng theo phối cảnh (góc quay camera) hoặc đặt lệch.
Do đó, OBB giúp:
\begin{itemize}
    \item bám sát hình dạng biển hiệu hơn BBox,
    \item giảm vùng nền bị crop dư thừa,
    \item cải thiện bước align/rectify và tăng chất lượng patch đưa vào text detector.
\end{itemize}

\subsubsection{Thiết lập fine-tune}
Trong thực nghiệm, các mô hình detection được fine-tune theo cùng một giao thức để đảm bảo so sánh công bằng:
\begin{itemize}
    \item Huấn luyện trong $50$ epochs.
    \item Áp dụng augmentation cơ bản cho bài toán street-view (phóng to/thu nhỏ, xoay nhẹ, thay đổi sáng/tương phản, motion blur mức nhẹ nếu cần).
    \item Tiêu chí chọn checkpoint dựa trên mAP (đặc biệt là AP50 và AP50--95).
\end{itemize}
Lưu ý: các siêu tham số chi tiết (batch size, lr, scheduler) có thể thay đổi tùy GPU; khóa luận tập trung mô tả pipeline và so sánh theo cùng một chuẩn đánh giá.

\subsection{Phân đoạn biển hiệu bằng segmentation}
Đối với các biển hiệu có hình dạng không đều hoặc bị che khuất một phần, bounding box có thể bao quá nhiều nền, làm giảm chất lượng text detection.
Vì vậy nhóm bổ sung hướng segmentation để dự đoán mask biển hiệu:
\[
f_{\text{seg}}(I) \rightarrow m(x,y)
\]
Sau đó, mask được dùng để:
\begin{itemize}
    \item crop theo vùng mask (tight crop),
    \item hoặc giữ nguyên kích thước ảnh nhưng xóa nền ngoài mask (background suppression).
\end{itemize}

\subsubsection{Tạo patch biển hiệu từ mask}
Từ mask $m$, nhóm lấy hộp bao nhỏ nhất (min bounding rectangle) hoặc OBB của vùng mask để crop ra patch biển hiệu $I_{sb}$.
Trong một số trường hợp, việc giữ mask để ``triệt nền'' giúp text detector ổn định hơn khi nền phía sau biển hiệu quá phức tạp.

% -------------------------------------------------
\section{Chuẩn hóa hình học (Align/Rectify)}

\subsection{Động cơ của bước Align}
Bước Align nhằm giảm tác động của phối cảnh và nghiêng xoay, giúp:
\begin{itemize}
    \item text detector phát hiện ổn định hơn (đặc biệt với chữ thẳng hàng),
    \item recognizer giảm lỗi do kí tự bị kéo dãn/biến dạng.
\end{itemize}

\subsection{Align dựa trên OBB/4 điểm}
Với OBB biểu diễn bởi 4 đỉnh $\{(x_k,y_k)\}_{k=1}^{4}$, nhóm thực hiện phép biến đổi phối cảnh (homography) để đưa biển hiệu (hoặc vùng chữ) về mặt phẳng chuẩn.
Khi đó:
\[
\hat{I} = \text{WarpPerspective}(I, \mathbf{H})
\]
với $\mathbf{H}$ là ma trận homography 3x3 ước lượng từ cặp điểm nguồn--đích.

\begin{figure}[t]
    \centering
    \includegraphics[width=0.98\linewidth]{align_rectify.png}
    \caption{Minh họa bước Align/Rectify dựa trên 4 điểm (OBB) để giảm nghiêng và phối cảnh trước khi phát hiện/nhận dạng chữ.}
    \label{fig:ch3_align_rectify}
\end{figure}

% -------------------------------------------------
\section{Mô-đun phát hiện văn bản (Text Detection)}

\subsection{Bài toán phát hiện văn bản trong patch biển hiệu}
Với patch biển hiệu $I_{sb}$, text detector dự đoán tập vùng chữ:
\[
f_{\text{textdet}}(I_{sb}) \rightarrow \mathcal{B}=\{(b^{(j)}, p^{(j)})\}_{j=1}^{N}
\]
Trong đó $b^{(j)}$ có thể là bbox/obb/polygon tùy mô hình; $p^{(j)}$ là độ tin cậy.

\subsection{Lựa chọn hướng tiếp cận: segmentation-based vs OBB-based}
Nhóm xét hai hướng chính phù hợp dữ liệu signboard:
\begin{itemize}
    \item \textbf{Segmentation-based (ví dụ: DBNet++, TextPMs):} linh hoạt với chữ cong, chữ dày/nhỏ, nhưng tách instance gần nhau có thể khó.
    \item \textbf{OBB-based (ví dụ: YOLOv8-obb, YOLOv11-obb):} dự đoán hộp quay trực tiếp, phù hợp chữ nghiêng và cho tốc độ cao.
\end{itemize}

\subsection{Quy trình fine-tune text detector}
Sau khi đánh giá các mô hình tiền huấn luyện, nhóm chọn các mô hình có kết quả tốt để fine-tune trên SignboardText.
Quy trình huấn luyện gồm:
\begin{itemize}
    \item Chuẩn hoá dữ liệu nhãn về định dạng của mô hình (bbox/obb hoặc polygon).
    \item Augmentation tập trung vào hiện tượng street-view: blur nhẹ, thay đổi ánh sáng, scale, rotate nhỏ.
    \item Hậu xử lý:
    \begin{itemize}
        \item NMS (hoặc NMS xoay cho OBB) để loại bỏ dự đoán trùng.
        \item Lọc theo confidence threshold để giảm false positives.
    \end{itemize}
\end{itemize}

\subsection{Chỉ số đánh giá cho text detection}
Khóa luận sử dụng Precision/Recall/Hmean:
\[
P = \frac{TP}{TP+FP},\quad
R = \frac{TP}{TP+FN},\quad
H = \frac{2PR}{P+R}
\]
Tiêu chí khớp (matching) giữa dự đoán và ground-truth dựa trên IoU (hoặc IoU cho OBB/polygon tuỳ dạng nhãn).

% -------------------------------------------------
\section{Mô-đun nhận dạng văn bản (Text Recognition)}

\subsection{Bài toán nhận dạng chuỗi ký tự}
Với mỗi vùng chữ $b^{(j)}$, ta crop/rectify được patch chữ $\hat{I}^{(j)}$.
Text recognizer dự đoán chuỗi:
\[
f_{\text{textrec}}(\hat{I}^{(j)}) \rightarrow (\mathbf{s}^{(j)}, c^{(j)})
\]
trong đó $\mathbf{s}^{(j)}$ là chuỗi ký tự, $c^{(j)}$ là độ tin cậy (hoặc xác suất trung bình theo token).

\subsection{Tiền xử lý patch chữ}
Để tăng độ ổn định cho recognizer, nhóm áp dụng các bước chuẩn hoá đầu vào:
\begin{itemize}
    \item Resize về kích thước chuẩn theo yêu cầu mô hình.
    \item Giữ tỉ lệ (aspect ratio) khi có thể; dùng padding để tránh méo chữ.
    \item (Tùy chọn) tăng tương phản cục bộ nhẹ cho chữ mờ/thiếu sáng.
\end{itemize}

\subsection{Mô hình nhận dạng và lý do lựa chọn}
Các mô hình recognizer hiện đại (đặc biệt transformer-based) phù hợp văn bản tự nhiên do khả năng học phụ thuộc dài và chống biến dạng tốt.
Trong khóa luận, nhóm đánh giá các mô hình tiền huấn luyện và chọn mô hình có hiệu quả tốt để fine-tune trên SignboardText (đặc biệt trên Vietsignboard và Vin).

\subsection{Chỉ số đánh giá cho text recognition}
Khóa luận sử dụng:
\begin{itemize}
    \item \textbf{Exact-match accuracy:} dự đoán đúng hoàn toàn chuỗi ký tự.
    \item \textbf{Normalized-match accuracy:} so khớp sau khi chuẩn hoá (ví dụ: chuẩn hoá Unicode/NFC, bỏ khoảng trắng dư, chuẩn hoá dấu câu, hoặc chuẩn hoá chữ hoa-thường theo quy ước).
\end{itemize}

% -------------------------------------------------
\section{Hậu xử lý và chuẩn hoá tiếng Việt}

\subsection{Chuẩn hoá Unicode và lọc nhiễu ký tự}
Văn bản tiếng Việt thường phát sinh lỗi về mã Unicode (tổ hợp dấu) hoặc ký tự nhiễu do nền phức tạp.
Do đó, nhóm thực hiện:
\begin{itemize}
    \item Chuẩn hoá Unicode về một chuẩn thống nhất (ví dụ NFC).
    \item Loại bỏ ký tự không hợp lệ theo tập ký tự mục tiêu (alphabet) của dữ liệu.
    \item Lọc dự đoán theo confidence; ưu tiên kết quả có độ tin cậy cao hơn trong các trường hợp trùng lặp vùng chữ.
\end{itemize}

\subsection{Gộp kết quả theo dòng/khối (tuỳ chọn)}
Trong trường hợp text detector trả về nhiều box theo từng từ/ký tự, có thể gộp theo dòng dựa trên:
\begin{itemize}
    \item độ gần theo trục ngang,
    \item độ chênh lệch góc quay,
    \item và khoảng cách giữa các hộp.
\end{itemize}
Bước này giúp tạo câu/nhãn biển hiệu hoàn chỉnh hơn (tùy yêu cầu đầu ra).

% -------------------------------------------------
\section{Tổng hợp cấu hình thực nghiệm trong pipeline}

\subsection{Nguyên tắc chọn mô hình cho pipeline cuối}
Từ đánh giá mô-đun (signboard det/seg, text det, text rec), nhóm chọn cấu hình kết hợp dựa trên:
\begin{itemize}
    \item \textbf{Độ chính xác:} ưu tiên Hmean cao cho text detection và accuracy cao cho recognition.
    \item \textbf{Tốc độ:} cân bằng FPS để phù hợp xử lý video.
    \item \textbf{Tính ổn định:} ưu tiên cấu hình ít nhạy với nền phức tạp và góc nhìn.
\end{itemize}

\subsection{Các biến thể pipeline được so sánh}
Khóa luận so sánh một số biến thể theo cấu trúc:
\[
(\text{Signboard module}) + (\text{Text detector}) + (\text{Text recognizer}) + (\text{Align})
\]
Trong đó:
\begin{itemize}
    \item Signboard module gồm: object detection (BBox/OBB) hoặc segmentation.
    \item Text detector gồm: segmentation-based hoặc OBB-based.
    \item Text recognizer: mô hình transformer-based fine-tune.
    \item Align: bật/tắt theo cấu hình để đánh giá tác động của chuẩn hoá hình học.
\end{itemize}

% -------------------------------------------------
\section{Tóm tắt chương}
Chương này đã trình bày phương pháp xây dựng hệ thống đọc chữ trên biển hiệu theo hướng pipeline module hoá.
Hệ thống gồm ba thành phần chính: phát hiện/phan đoạn biển hiệu để giới hạn vùng quan tâm, phát hiện văn bản trong patch biển hiệu (ưu tiên OBB/segmentation tuỳ bối cảnh), và nhận dạng chuỗi ký tự bằng mô hình recognizer hiện đại.
Ngoài ra, chương cũng mô tả bước Align/Rectify để giảm méo hình học, cùng các bước hậu xử lý và chuẩn hoá tiếng Việt nhằm tăng độ ổn định đầu ra.
Các cấu hình pipeline khác nhau sẽ được đánh giá chi tiết bằng thực nghiệm trong chương tiếp theo.
