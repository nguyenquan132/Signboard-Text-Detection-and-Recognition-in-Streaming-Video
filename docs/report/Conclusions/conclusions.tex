% =========================
% CHAPTER 6: KẾT LUẬN VÀ HƯỚNG PHÁT TRIỂN
% =========================
\chapter{KẾT LUẬN VÀ HƯỚNG PHÁT TRIỂN}
\ifpdf
    \graphicspath{{Chapter6/Chapter6Figs/PNG/}{Chapter6/Chapter6Figs/PDF/}{Chapter6/Chapter6Figs/}}
\else
    \graphicspath{{Chapter6/Chapter6Figs/EPS/}{Chapter6/Chapter6Figs/}}
\fi

\markboth{\MakeUppercase{\thechapter. KẾT LUẬN VÀ HƯỚNG PHÁT TRIỂN}}{\thechapter. KẾT LUẬN VÀ HƯỚNG PHÁT TRIỂN}

% -------------------------------------------------
\section{Kết luận}
% Trước nhu cầu gia tăng về việc tự động trích xuất thông tin từ ảnh và video trong môi trường đường phố,
% bài toán phát hiện và nhận dạng văn bản trên biển hiệu mang ý nghĩa thiết thực cho các ứng dụng như phân tích đô thị,
% bản đồ số theo ngữ nghĩa và hỗ trợ truy xuất thông tin theo địa điểm. Xuất phát từ bối cảnh đó, khóa luận đã tập trung
% nghiên cứu và phát triển một hệ thống ``đọc chữ trên biển hiệu'' theo hướng end-to-end, phù hợp với dữ liệu street-view và video hành trình.

Trước nhu cầu ngày càng tăng trong việc khai thác thông tin từ cảnh quan đô thị, biển hiệu trở thành nguồn dữ liệu văn bản giàu ngữ nghĩa, cung cấp thông tin quan trọng cho các hệ thống thị giác máy tính. Chính vì vậy, việc tự động phát hiện và trích xuất nội dung từ các biển hiệu đóng vai trò quan trọng trong việc chuyển đổi dữ liệu thô thành tri thức có thể khai thác, từ đó hỗ trợ các nhiệm vụ trích xuất thông tin một cách hiệu quả. Trên cơ sở đó, khóa luận tập trung phát triển một pipeline đầu-cuối (end-to-end), cung cấp đồng thời thông tin về vị trí và nội dung văn bản trên biển hiệu.

Xuất phát từ mục tiêu trên, ba đóng góp chính của khóa luận bao gồm:
\begin{itemize}
    \item \textbf{Mở rộng bộ dữ liệu:} Bổ sung nhãn đối tượng biển hiệu (signboard bounding box) cho tập dữ liệu ảnh tĩnh SignboardText, đồng thời thu thập một tập dữ liệu video hành trình thực tế, phục vụ minh họa và kiểm tra tính tổng quát của pipeline.
    \item \textbf{Thực nghiệm và phân tích:} Cung cấp các thực nghiệm và phân tích chi tiết, làm cơ sở cho các nghiên cứu và ứng dụng trong tương lai.
    \item \textbf{Phát triển pipeline đầu-cuối (end-to-end):} Xây dựng pipeline hoàn chỉnh cho bài toán phát hiện và nhận dạng văn bản trên biển hiệu trong video hành trình tại Việt Nam.
\end{itemize}

Hiệu quả của pipeline đề xuất đã được đánh giá trên tập SignboardText, với khả năng xử lý tốt các biển hiệu chứa phần lớn ngôn ngữ Tiếng Việt. Nhờ đó, pipeline không chỉ hỗ trợ các ứng dụng trích xuất thông tin trong thực tế mà còn cung cấp công cụ hữu ích cho việc thu thập và quản lý dữ liệu trong lĩnh vực thị giác máy tính. Tuy nhiên, trong quá trình thực nghiệm và đánh giá, khóa luận nhận thấy một số hạn chế đáng lưu ý:

\begin{itemize}
    \item \textbf{Điều kiện môi trường:} Thực nghiệm chủ yếu được thực hiện trong các điều kiện thuận lợi, do đó hiệu quả của pipeline trong các môi trường bất lợi như mưa, sương mù hoặc ánh sáng yếu vẫn chưa được kiểm chứng đầy đủ.
    \item \textbf{Dữ liệu đánh giá:} Mặc dù bước tiền xử lý căn chỉnh biển hiệu giúp cải thiện hiệu quả pipeline với mô hình phát hiện biển hiệu SegFormer, tập dữ liệu SignboardText vẫn còn hạn chế về số lượng biển hiệu với các góc nghiêng khác nhau, làm hạn chế khả năng đánh giá toàn diện hiệu quả pipeline.
\end{itemize}



% Về mặt phương pháp, khóa luận tiếp cận bài toán theo hướng \textbf{pipeline module hoá} gồm ba thành phần chính:
% (i) phát hiện/phân đoạn biển hiệu để giới hạn vùng quan tâm và giảm nhiễu nền,
% (ii) phát hiện vùng chữ bên trong biển hiệu bằng các biểu diễn hình học phù hợp (đặc biệt hữu ích khi chữ bị nghiêng),
% và (iii) nhận dạng chuỗi ký tự từ các vùng chữ đã crop/rectify.
% Trên cơ sở bộ dữ liệu SignboardText, khóa luận đã chuẩn bị dữ liệu cho các tác vụ, thực nghiệm so sánh nhiều mô hình hiện đại cho từng mô-đun,
% và đánh giá các cấu hình pipeline nhằm lựa chọn phương án phù hợp.

% Kết quả thực nghiệm ở Chương~4 cho thấy chiến lược ``phát hiện biển hiệu trước'' giúp tăng tính ổn định và giảm nhiễu nền so với xử lý trực tiếp trên toàn ảnh,
% đồng thời các lựa chọn biểu diễn hình học (OBB/đa giác) và bước chuẩn hóa hình học (Align/Rectify) có đóng góp tích cực trong các trường hợp phối cảnh/ nghiêng.
% Dựa trên các kết quả này, hệ thống đã được triển khai thành ứng dụng minh họa (Chương~5), qua đó xác nhận tính khả dụng của pipeline khi áp dụng lên dữ liệu video.

% Các đóng góp chính của khóa luận được tóm tắt như sau:
% \begin{itemize}
%     \item \textbf{Chuẩn bị và mở rộng dữ liệu}: tổng hợp dữ liệu và chuẩn hóa nhãn phục vụ thực nghiệm cho các tác vụ phát hiện biển hiệu, phát hiện chữ và nhận dạng chữ.
%     \item \textbf{Thực nghiệm và đánh giá}: triển khai so sánh mô hình theo mô-đun và theo pipeline, kèm phân tích ưu/nhược điểm trong bối cảnh biển hiệu tiếng Việt và môi trường đường phố.
%     \item \textbf{Xây dựng hệ thống end-to-end}: đề xuất và triển khai pipeline tích hợp, cung cấp đầu ra gồm vị trí biển hiệu, vị trí vùng chữ và nội dung văn bản nhận dạng.
%     \item \textbf{Ứng dụng minh họa}: xây dựng ứng dụng xử lý ảnh/video với khả năng trực quan hóa và xuất kết quả, tạo nền tảng cho các tác vụ khai thác thông tin về sau.
% \end{itemize}

% Bên cạnh những kết quả đạt được, khóa luận vẫn tồn tại một số hạn chế:
% \begin{itemize}
%     \item \textbf{Độ nhạy với chất lượng đầu vào}: chất lượng phát hiện/nhận dạng suy giảm khi video bị mờ do chuyển động, rung mạnh, thiếu sáng hoặc chữ quá nhỏ.
%     \item \textbf{Sai số lan truyền theo pipeline}: lỗi từ bước phát hiện/phân đoạn biển hiệu có thể ảnh hưởng đến toàn bộ các bước sau, đặc biệt trong trường hợp biển hiệu bị che khuất hoặc nền phức tạp.
%     \item \textbf{Chưa khai thác thông tin thời gian}: hệ thống hiện tại chủ yếu xử lý theo từng khung hình độc lập nên chưa tận dụng tính dư thừa theo thời gian để tăng độ ổn định.
% \end{itemize}

% -------------------------------------------------
\section{Hướng phát triển}
Nhằm khắc phục những hạn chế đã nêu và nâng cao hiệu quả của pipeline trong các điều kiện thực tế đa dạng, khóa luận đề xuất một số hướng phát triển trong tương lai như sau: 

\begin{itemize}
\item \textbf{Cải thiện dữ liệu:} Xây dựng bộ dữ liệu chuyên biệt chứa các biển hiệu nghiêng ở nhiều góc độ khác nhau, nhằm hỗ trợ việc đánh giá và cải thiện bước căn chỉnh biển hiệu, từ đó góp phần nâng cao hiệu quả pipeline.
\item \textbf{Mở rộng điều kiện:} Thực nghiệm pipeline trong các môi trường bất lợi, bao gồm mưa, ánh sáng yếu và sương mù, để kiểm chứng tính ổn định và khả năng triển khai thực tế.
\item \textbf{Ứng dụng thực tế:} Phát triển giao diện người dùng cho ứng dụng phát hiện và nhận dạng văn bản trên biển hiệu, hỗ trợ phân tích biển hiệu, tra cứu địa điểm và thông tin cửa hàng tại Việt Nam.
\end{itemize}

% \subsection{Khai thác thông tin thời gian trong video}
% \begin{itemize}
%     \item \textbf{Theo dõi biển hiệu/vùng chữ}: áp dụng tracking để gắn ID cho biển hiệu qua nhiều khung hình, giảm tính ngẫu nhiên và ổn định kết quả theo thời gian.
%     \item \textbf{Gộp kết quả theo thời gian (temporal fusion)}: hợp nhất nhiều kết quả nhận dạng của cùng một biển hiệu dựa trên độ tin cậy hoặc cơ chế bỏ phiếu, giúp giảm lỗi do mờ/nhiễu ở từng frame.
% \end{itemize}

% \subsection{Tối ưu hóa triển khai và hiệu năng suy luận}
% \begin{itemize}
%     \item \textbf{Tối ưu mô hình và suy luận}: tối ưu hóa tốc độ để đáp ứng bài toán video, bao gồm tối ưu tính toán, giảm độ trễ và cải thiện throughput khi mật độ biển hiệu/vùng chữ cao.
%     \item \textbf{Chiến lược xử lý thích ứng}: điều chỉnh động các ngưỡng lọc, cơ chế chọn vùng quan tâm và quy mô xử lý theo độ phức tạp của cảnh.
% \end{itemize}

% \subsection{Nâng cao độ bền vững với điều kiện khó}
% \begin{itemize}
%     \item \textbf{Mở rộng dữ liệu và tăng cường theo miền}: bổ sung kịch bản ban đêm, thời tiết xấu, camera chất lượng thấp; tăng cường dữ liệu mô phỏng motion blur, glare và nén video.
%     \item \textbf{Cải tiến chuẩn hóa hình học}: nâng cao chất lượng crop/rectify với các trường hợp phối cảnh mạnh, biển hiệu cong/méo hoặc bị che khuất một phần.
% \end{itemize}

% \subsection{Khai thác thông tin sau nhận dạng}
% \begin{itemize}
%     \item \textbf{Hậu xử lý tiếng Việt theo ngữ cảnh}: tích hợp từ điển miền hoặc mô hình ngôn ngữ để sửa lỗi dấu/ký tự dễ nhầm và chuẩn hóa văn bản đầu ra.
%     \item \textbf{Truy xuất và phân tích ở mức ứng dụng}: xây dựng chức năng tìm kiếm theo từ khóa, thống kê loại hình dịch vụ theo khu vực, làm nền tảng cho các ứng dụng bản đồ số và phân tích đô thị.
% \end{itemize}

% \subsection{Chuẩn hóa quy trình đánh giá trên video}
% \begin{itemize}
%     \item \textbf{Xây dựng bộ kiểm thử video có gán nhãn}: thiết kế benchmark ở mức frame/track giúp đánh giá công bằng các hướng temporal tracking/fusion.
%     \item \textbf{Đánh giá toàn diện theo kịch bản}: bổ sung tiêu chí về độ ổn định theo thời gian, độ trễ và khả năng đáp ứng trong môi trường triển khai thực.
% \end{itemize}

% % -------------------------------------------------
% \section{Tóm tắt chương}
% Chương này đã tổng kết các kết quả chính của khóa luận, bao gồm đóng góp về dữ liệu, thực nghiệm và xây dựng pipeline end-to-end cho bài toán đọc chữ trên biển hiệu.
% Đồng thời, chương cũng chỉ ra các hạn chế còn tồn tại và đề xuất các hướng phát triển tập trung vào khai thác thông tin thời gian trong video,
% tối ưu triển khai, tăng độ bền vững trước nhiễu và mở rộng khả năng khai thác thông tin sau nhận dạng.