\chapter{CƠ SỞ LÝ THUYẾT VÀ CÁC NGHIÊN CỨU LIÊN QUAN}

\ifpdf
    \graphicspath{{Chapter2/Chapter2Figs/PNG/}{Chapter2/Chapter2Figs/PDF/}{Chapter2/Chapter2Figs/Chapter2Figs/}}
\else
    \graphicspath{{Chapter2/Chapter2Figs/EPS/}{Chapter2/Chapter2Figs/}}
\fi

\markboth{\MakeUppercase{\thechapter. Cơ sở lý thuyết và các nghiên cứu liên quan}}{\thechapter. Cơ sở lý thuyết và các nghiên cứu liên quan}

\section{Giới thiệu}

\subsection{Tổng quan và ý nghĩa thực tiễn của bài toán đọc văn bản trên biển hiệu trong ảnh/video đường phố}
Trong môi trường giao thông đô thị, biển hiệu và bảng quảng cáo (signboards) là nguồn thông tin quan trọng phản ánh danh tính địa điểm (tên cửa hàng), cũng như loại sản phẩm/dịch vụ mà địa điểm đó cung cấp.
Với dữ liệu video quay từ camera hành trình, hệ thống ``đọc văn bản trong cảnh'' (scene text reading) có thể hỗ trợ nhiều ứng dụng thực tế như:
(i) lập bản đồ/định vị theo ngữ nghĩa (semantic mapping), (ii) tìm kiếm địa điểm theo từ khóa (place search),
(iii) thống kê loại hình kinh doanh theo khu vực, và (iv) hỗ trợ nhận thức tình huống trong các hệ thống giao thông thông minh.

Bài toán trong khóa luận tập trung vào xây dựng một pipeline tích hợp gồm:
\begin{itemize}
    \item \textbf{Phát hiện (Text Detection):} xác định và khoanh vùng các vùng chứa văn bản trong từng khung hình.
    \item \textbf{Nhận dạng (Text Recognition):} chuyển đổi ảnh vùng chữ thành chuỗi ký tự.
\end{itemize}

\subsection{Thách thức về tính đa dạng và phức tạp của văn bản trong môi trường tự nhiên}
Khác với tài liệu quét (scanned documents), văn bản trong cảnh đường phố thường xuất hiện trong điều kiện chụp không kiểm soát.
Các thách thức nổi bật bao gồm:
\begin{itemize}
    \item \textbf{Nền phức tạp (cluttered background):} nhiều vật thể/hoa văn gây nhiễu.
    \item \textbf{Văn bản biến dạng:} chữ cong/nghiêng/méo do phối cảnh, bề mặt biển hiệu hoặc góc nhìn.
    \item \textbf{Đa dạng phông chữ và kích thước:} font stylized, độ dày nét khác nhau, chữ rất nhỏ hoặc rất lớn.
    \item \textbf{Đa ngôn ngữ và dấu:} tiếng Việt có dấu, có thể xen kẽ tiếng Anh/Trung/Hàn; dấu câu đa dạng.
    \item \textbf{Motion blur và out-of-focus:} do xe di chuyển, rung camera, tốc độ cao.
    \item \textbf{Độ phân giải thấp (low resolution):} chữ nhỏ, ở xa camera, nén video làm mất chi tiết.
\end{itemize}

\subsubsection{Bài toán Scene Text Detection and Recognition}
Tổng quát, bài toán Scene Text Reading có thể được tiếp cận theo ba hướng chính:
\begin{enumerate}
    \item \textbf{Text Detection:} chỉ dự đoán vị trí vùng chữ (bounding box / polygon / mask).
    \item \textbf{Text Recognition:} nhận dạng ký tự/chuỗi ký tự từ các vùng chữ đã được cắt sẵn.
    \item \textbf{End-to-End Text Spotting/Recognition:} kết hợp phát hiện và nhận dạng trong một pipeline thống nhất.
\end{enumerate}
Trong khóa luận, trọng tâm là xây dựng pipeline tích hợp cho dữ liệu street-view/video, ưu tiên nhóm đối tượng biển hiệu/bảng quảng cáo cửa hàng.

\subsubsection{Phân loại hướng tiếp cận cho bài toán Text Detection and Recognition}
Các phương pháp học sâu cho bài toán đọc văn bản trong ảnh ngoại cảnh (scene text reading) thường được phân nhóm theo phạm vi xử lý:
(i) chỉ phát hiện vùng chữ, (ii) chỉ nhận dạng chữ trên vùng cắt sẵn, hoặc (iii) pipeline end-to-end kết hợp cả hai.

\begin{figure}[t]
    \centering
    \includegraphics[width=0.98\linewidth]{text_detection_recognition_taxonomy.png}
    \caption{Phân nhóm các phương pháp cho bài toán Text Detection and Recognition trong ảnh ngoại cảnh.}
    \label{fig:text_detection_recognition_taxonomy}
\end{figure}

\paragraph{Text Detection.}
\begin{itemize}
    \item \textbf{Regression-based:} hồi quy trực tiếp hộp/tứ giác bao quanh vùng chữ.
    \item \textbf{Connected component-based:} phát hiện thành phần ký tự (hoặc stroke) rồi liên kết thành dòng/từ.
    \item \textbf{Segmentation-based:} dự đoán bản đồ pixel thuộc text, sau đó tách instance bằng hậu xử lý.
\end{itemize}

\paragraph{Text Recognition.}
\begin{itemize}
    \item \textbf{Segmentation-based:} tách ký tự (hoặc vùng con) rồi nhận dạng.
    \item \textbf{Segmentation-free:} nhận dạng trực tiếp chuỗi (CTC/attention/transformer) không cần tách ký tự.
\end{itemize}

\paragraph{End-to-End Text Recognition.}
\begin{itemize}
    \item \textbf{One-stage:} phát hiện và nhận dạng trong một mô hình thống nhất.
    \item \textbf{Two-stage:} phát hiện trước, sau đó cắt/chuẩn hóa và nhận dạng ở mô-đun thứ hai.
\end{itemize}


\section{Cơ sở lý thuyết}

\subsection{Biểu diễn dữ liệu video và trích xuất khung hình}
Video được xem như chuỗi khung hình (frame) theo thời gian.
Cho video $V$, ta trích xuất tập khung hình $\{I_t\}_{t=1}^{T}$ với tốc độ lấy mẫu phù hợp (ví dụ: lấy mọi frame hoặc lấy theo bước nhảy để tối ưu tính toán).
Vì văn bản có thể xuất hiện trong nhiều frame liên tiếp, dữ liệu video mang \textit{tính dư thừa theo thời gian} (temporal redundancy) có thể khai thác để tăng độ ổn định.

\subsection{Quy trình xử lý tổng thể (Integrated Pipeline)}
Pipeline đề xuất ở mức khái niệm gồm các bước:
\begin{enumerate}
    \item \textbf{Tiền xử lý (Pre-processing):}
    \begin{itemize}
        \item Giảm nhiễu, cân bằng sáng, tăng tương phản cục bộ khi cần thiết.
        \item Giảm mờ do chuyển động (deblurring) hoặc tăng độ phân giải (super-resolution) cho vùng chữ nhỏ (tùy tài nguyên).
        \item Ổn định video (video stabilization) trong trường hợp rung mạnh.
    \end{itemize}

    \item \textbf{Phát hiện vùng văn bản (Text Detection):}
    Dự đoán vị trí vùng chữ theo dạng hộp (box), tứ giác (quadrilateral), đa giác (polygon) hoặc mặt nạ (segmentation mask).
    Đầu ra gồm tập vùng $\mathcal{B}_t = \{b_{t}^{(i)}\}$ tại frame $I_t$.

    \item \textbf{Chuẩn hóa hình học và cắt vùng chữ (Crop \& Rectify):}
    Với vùng chữ nghiêng/cong, cần biến đổi phối cảnh hoặc chuẩn hóa hình học để đưa về ảnh chữ ``thẳng'' (rectified) trước khi nhận dạng.
    Gọi ảnh vùng chữ sau chuẩn hóa là $\hat{I}_{t}^{(i)}$.

    \item \textbf{Nhận dạng văn bản (Text Recognition):}
    Mô hình nhận dạng thực hiện ánh xạ $\hat{I}_{t}^{(i)} \rightarrow \mathbf{s}_{t}^{(i)}$,
    trong đó $\mathbf{s}_{t}^{(i)}$ là chuỗi ký tự dự đoán.

    \item \textbf{Hậu xử lý (Post-processing):}
    \begin{itemize}
        \item Chuẩn hóa Unicode tiếng Việt, sửa lỗi dấu/telex nếu cần.
        \item Loại bỏ ký tự nhiễu, lọc theo độ tin cậy (confidence).
        \item Gộp các kết quả theo thời gian (temporal fusion) nếu cùng một biển hiệu xuất hiện ở nhiều frame.
    \end{itemize}

    \item \textbf{Suy luận loại dịch vụ/sản phẩm (Semantic Inference):}
    Từ chuỗi ký tự $\mathbf{s}$, hệ thống gán nhãn ngành hàng/dịch vụ bằng:
    (i) luật từ khóa (keyword rules), (ii) phân lớp văn bản (text classification), hoặc (iii) kết hợp NER + taxonomy.
\end{enumerate}

\subsection{Cơ sở lý thuyết Text Detection}
Text Detection trong cảnh có thể chia thành các hướng chính:
\begin{itemize}
    \item \textbf{Regression-based:} dự đoán trực tiếp hộp/tứ giác bao quanh text.
    \item \textbf{Connected-component based:} phát hiện thành phần ký tự và liên kết thành cụm.
    \item \textbf{Segmentation-based:} dự đoán bản đồ pixel thuộc vùng text và tách instance bằng hậu xử lý.
\end{itemize}
Trong thực tế signboards, hướng segmentation-based phổ biến vì linh hoạt với chữ cong/biến dạng, nhưng gặp khó khi các cụm chữ nằm gần nhau gây chồng lấp.

\subsection{Cơ sở lý thuyết Text Recognition}
Text Recognition thường được mô hình hóa như bài toán nhận dạng chuỗi:
\begin{itemize}
    \item \textbf{CTC-based:} ánh xạ đặc trưng theo chiều ngang thành chuỗi ký tự với CTC loss.
    \item \textbf{Attention/Encoder-Decoder:} sinh chuỗi theo cơ chế chú ý.
    \item \textbf{Transformer-based recognizer:} tận dụng self-attention để học phụ thuộc dài và chống méo tốt hơn.
\end{itemize}
Với tiếng Việt, các yếu tố dấu và biến thể font làm tăng độ khó; hậu xử lý chuẩn hóa và từ điển miền (domain lexicon) có thể cải thiện độ chính xác.

\subsection{Khai thác thông tin thời gian trong video}
Khác ảnh tĩnh, video cho phép:
\begin{itemize}
    \item \textbf{Tracking text regions:} theo dõi một vùng chữ qua nhiều frame, giảm số lần chạy recognizer (nhận dạng một lần cho cả track).
    \item \textbf{Temporal fusion:} hợp nhất nhiều kết quả nhận dạng theo vote/confidence/edit distance để tăng độ ổn định.
\end{itemize}

\section{Các nghiên cứu liên quan}

\subsection{Bộ dữ liệu văn bản trong video lái xe: RoadText-1K}
RoadText-1K giới thiệu bộ dữ liệu lớn cho bài toán phát hiện và nhận dạng văn bản trong video lái xe, gồm các đoạn clip được lấy mẫu từ dữ liệu lái xe thực tế và được gán nhãn dày (dense) theo từng frame.
Bộ dữ liệu cung cấp:
(i) bounding boxes cho vùng chữ, (ii) phiên âm (transcription), và (iii) nhãn phân loại text (ví dụ: tiếng Anh/không phải tiếng Anh/không đọc được), đồng thời tách riêng trường hợp biển số xe trong nhóm tiếng Anh.
RoadText-1K được thiết kế theo hướng ``không thiên lệch theo text'' (unconstrained), phản ánh đúng bối cảnh camera hành trình với các nhiễu như motion blur, out-of-focus và glare.
Các đánh giá baseline cho thấy các phương pháp SOTA trên ảnh tĩnh khi áp dụng vào video lái xe sẽ gặp suy giảm đáng kể do độ khó tăng lên.

\subsection{Phương pháp phát hiện text theo cơ chế ``spotlight'': Spotlight Text Detector (STD)}
Spotlight Text Detector (STD) tập trung giải quyết hai vấn đề lớn của text detection dạng segmentation:
(i) các instance chữ nằm gần nhau gây chồng lấp khó tách, và (ii) hình dạng/độ dài chữ biến thiên lớn khiến mô hình khó khái quát.
STD đề xuất hai thành phần chính:
\begin{itemize}
    \item \textbf{Spotlight Calibration Module (SCM):} hiệu chỉnh vùng ứng viên (candidate kernel) dựa trên coarse mask, tương tự cơ chế camera ``focus'' vào mục tiêu; module này giúp giảm false positives bằng cách hiệu chỉnh dự đoán và tăng khả năng tập trung vào vùng kernel quan trọng.
    \item \textbf{Multivariate Information Extraction Module (MIEM):} trích xuất thông tin hình học đa dạng theo nhiều ``shape schemes'', nhằm học tốt hơn các đặc trưng tỷ lệ, hướng và hình dạng của chữ trong cảnh.
\end{itemize}
Kết quả thực nghiệm cho thấy STD đạt hiệu năng cạnh tranh/vượt trội trên nhiều benchmark text detection phổ biến (ICDAR2015, CTW1500, MSRA-TD500, Total-Text), đồng thời ablation chứng minh đóng góp của SCM và MIEM.

\subsection{Liên hệ với đề tài khóa luận}
Từ các nghiên cứu trên, có thể rút ra các định hướng quan trọng cho bài toán biển hiệu trong video street-view:
\begin{itemize}
    \item \textbf{Về dữ liệu và đánh giá:} cần ưu tiên bối cảnh ``unconstrained driving video'' và các dạng nhiễu đặc thù; RoadText-1K là nguồn tham khảo về cách thiết kế dữ liệu/nhãn và tiêu chí benchmark.
    \item \textbf{Về phát hiện văn bản:} các phương pháp segmentation nâng cao cơ chế hiệu chỉnh/khoanh vùng (như SCM của STD) hữu ích khi text gần nhau và nền phức tạp --- đặc trưng thường gặp ở biển hiệu phố.
    \item \textbf{Về pipeline tích hợp:} cần kết hợp (i) detection mạnh với chữ cong/biến dạng, (ii) rectify hợp lý trước recognition, và (iii) cơ chế temporal fusion/tracking để ổn định kết quả trên video.
\end{itemize}

\section{Tóm tắt chương}
Chương này đã trình bày tổng quan bài toán đọc văn bản trên biển hiệu trong video đường phố, các thách thức đặc thù, cơ sở lý thuyết của hai thành phần chính (text detection và text recognition), cùng khả năng khai thác thông tin thời gian trong video.
Ngoài ra, chương cũng tổng hợp hai hướng nghiên cứu liên quan tiêu biểu: bộ dữ liệu RoadText-1K cho video lái xe và phương pháp Spotlight Text Detector cho phát hiện chữ với cơ chế hiệu chỉnh vùng ứng viên.
Những nội dung này là cơ sở để thiết kế pipeline tích hợp và xây dựng thực nghiệm trong các chương tiếp theo.
