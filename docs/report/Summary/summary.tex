\chapter*{\centering{TÓM TẮT KHÓA LUẬN}}
\addcontentsline{toc}{chapter}{Tóm tắt khóa luận}

Phát hiện và nhận dạng văn bản trong ảnh ngoại cảnh là một bài toán quan trọng trong thị giác máy tính, hướng đến mục tiêu khai thác thông tin ngữ nghĩa phục vụ nhiều ứng dụng thực tiễn, chẳng hạn như hỗ trợ dẫn đường thông minh và phân tích loại hình kinh doanh đô thị. Trong đó, văn bản trên biển hiệu đóng vai trò đặc biệt quan trọng do thường chứa các thông tin mang tính định danh như tên địa điểm, cơ sở kinh doanh và loại hình dịch vụ. Tuy nhiên, bài toán phát hiện và nhận dạng văn bản trên biển hiệu nói chung đặt ra nhiều thách thức, xuất phát từ sự đa dạng về hình thức và bố cục của văn bản, cũng như đặc điểm phức tạp của biển hiệu trong môi trường thực tế. Khi mở rộng từ ảnh tĩnh sang dữ liệu video hành trình, các yếu tố như chuyển động của phương tiện, chất lượng hình ảnh hạn chế và sự biến đổi liên tục của điều kiện quan sát càng làm gia tăng độ phức tạp của bài toán. Đặc biệt, trong bối cảnh đường phố Việt Nam, những thách thức này trở nên rõ rệt hơn do đặc trưng của tiếng Việt với hệ thống dấu thanh và các ký tự mở rộng, gây khó khăn cho cả phát hiện lẫn nhận dạng văn bản. Mặc dù vậy, các hệ thống tìm kiếm và phân tích loại hình kinh doanh ngày càng có nhu cầu khai thác thông tin ngữ nghĩa từ văn bản trên biển hiệu xuất hiện trong ảnh và video đường phố. 

Xuất phát từ nhu cầu đó, khóa luận này tập trung xây dựng một quy trình xử lý đầu-cuối (pipeline end-to-end) cho bài toán phát hiện và nhận dạng văn bản trên biển hiệu trong ảnh và video được ghi lại bởi camera hành trình. Quy trình xử lý (pipeline) đề xuất được thiết kế gồm ba giai đoạn chính, tương ứng với các tác vụ: phát hiện biển hiệu, phát hiện văn bản trong vùng biển hiệu và nhận dạng nội dung văn bản. Trên cơ sở khảo sát và thực nghiệm so sánh các phương pháp tiên tiến hiện nay cho từng tác vụ con, khóa luận tiến hành lựa chọn mô hình tối ưu và tích hợp chúng vào một quy trình xử lý (pipeline) thống nhất, nhằm đạt được sự cân bằng giữa độ chính xác và tính ổn định trong bối cảnh dữ liệu thực tế.

Các thực nghiệm được thực hiện trên tập dữ liệu SignboardText mở rộng, với đánh giá hiệu suất theo cách tiếp cận đầu-cuối (end-to-end) trên các tập con chứa văn bản tiếng Việt và tiếng Anh. Nhằm đánh giá hiệu suất trong bối cảnh mục tiêu là đường phố Việt Nam, phân tích được tập trung trên hai tập con tiếng Việt. Kết quả thực nghiệm cho thấy quy trình xử lý (pipeline) kết hợp RTDETRv2 cho phát hiện biển hiệu, YOLOv8-OBB cho phát hiện văn bản và PARSeq cho nhận dạng văn bản đạt chỉ số $Hmean$ trong giai đoạn phát hiện văn bản lần lượt là 89.64\% trên VietSignboard và 89.79\% trên VinText. Đồng thời, hiệu suất nhận dạng văn bản theo cách tiếp cận đầu-cuối (end-to-end) đạt chỉ số $Hmean_{e2e}$ tương ứng là 72.32\% và 72.23\%. Các kết quả này cho thấy tính hiệu quả của quy trình xử lý đầu-cuối (pipeline end-to-end) đề xuất. Bên cạnh đó, việc khảo sát bước căn chỉnh biển hiệu cho thấy tiềm năng cải thiện hiệu suất trong các trường hợp biển hiệu có góc nghiêng lớn hoặc hình dạng không chuẩn.

Tóm lại, những đóng góp chính của khóa luận bao gồm xây dựng một quy trình xử lý đầu-cuối (pipeline end-to-end) cho bài toán phát hiện và nhận dạng văn bản trên biển hiệu trong video hành trình, hướng tới bối cảnh đường phố Việt Nam, đồng thời cung cấp các kết quả và phân tích thực nghiệm có giá trị, làm cơ sở cho các nghiên cứu và ứng dụng tiếp theo trong lĩnh vực trích xuất thông tin từ ảnh và video ngoại cảnh.